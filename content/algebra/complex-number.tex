


\section{复数}
\label{sec:complex-number}

\subsection{复数的引入}
\label{sec:import-complex-number}



数系的扩充与解方程密切相关,为了解一元一次方程引入了有理数,数系被扩充为有理数集,为了解二次方程又发现了无理数,由此我们的数系被扩充到了实数集。但在实数范围内,方程$x^2+1=0$仍然无解,于是引入了虚数,虚数与实数一起构成了更加广阔的复数集。

复数集的引入,理由看似比较牵强,但后来的事实表明,很多数学理论在复数范围内都能得到完美的解决,例如代数学基本定理就表明,任何一个关于某未知数的$n$次方程,在复数范围内都有且仅有$n$个根(重根按重数计数),又例如,有些数列,它的每一项都是整数,然而它的通项,却只能借助复数来表达,又比如,在复数范围内,指数函数将与三角函数产生密切关系,欧拉公式$e^{ix}=\cos{x}+i\sin{x}$便揭示了这一点,而三角函数与双曲函数拥有许多类似的性质,但在实数范围内不容易看出它们有什么内在联系,而在复变函数领域内将会看到这并不是巧合。诸如此类,许多理论都表明复数具有重大的理论意义,这一点随着我们对高等数学的更多了解,将能有更多体会。

为了解方程$x^2+1=0$,引入一个数$i$,规定$i^2=-1$,因为实数范围内是不可能有某个数的平方是负的,所以这引入的$i$便称为一个\emph{虚数},并且称它是\emph{虚数单位}。规定虚数可以与实数相加与相乘,并且符合实数运算所满足的交换律、结合律、分配律。

那么将实数与虚数进行混合加法与乘法运算,会有什么结果呢,将$i$与实数$b$相乘得出$bi$,再将$bi$与实数$a$相加得到$a+bi$,这个形式无法继续化简了,它就是复数的一般形式,也就是任意一个复数都具有这种形式,我们就来证明它。

\begin{theorem}
  将实数与虚数单位$i$进行有限次加法与乘法的混合运算,得出的结果都具有形式$a+bi(a,b\in R)$.
\end{theorem}

\begin{proof}[证明]
  只利用加法与乘法,运算对象为实数与虚数单位$i$,那么最终结果是关于$i$的实系数多项式
  \[ a_ni^n+a_{n-1}i^{n-1} + \cdots + a_2i^2 + a_1i+a_0 \]
  根据虚数单位$i$的定义,可知其乘幂$i^n$依次循环取值$i,-1,-i,1$,由此便知定理成立。
\end{proof}

由此,复数都具有形式$z=a+bi(a,b\in R)$,$a$称为它的\emph{实部},记作$Re(z)$,$b$称为它的虚部,记作$Im(z)$,显然复数$a+bi$与数对$(a,b)$一一对应,于是便与坐标平面上的点一一对应,于是坐标平面上的每一个点都对应着一个复数,于是这平面便被称为\emph{复平面},而虚数单位 $i$ 对应着复平面上的点 $(0,1)$.

规定,两个复数相等当且仅当它们的实部和虚部分别相等.此外,复数集上没有大小关系。

称 $\sqrt{a^2+b^2}$ 为复数 $z=a+bi$ 的 \emph{模},记作 $|z|$,即 $|z|=\sqrt{a^2+b^2}$.

称复数$a-bi$为复数$z=a+bi$ 的 \emph{共轭复数},记作$\overline{z}$,显然,$z$的共轭复数的共轭复数是 $z$,因此$a+bi$与$a-bi$互为共轭,也就是 $\overline{\overline{z}}=z$.


\subsection{复数的运算}
\label{sec:operation-of-complex-number}

设 $z_1=a_1+b_1i$, $z_2=a_2+b_2i$,有加法运算
\[ z_1+z_2 = (a_1+b_1i)+(a_2+b_2i) = (a_1+a_2)+(b_1+b_2)i \]
即实部之和为和数的实部,虚部之和为和数的虚部.

同样有减法运算
\[ z_1-z_2 = (a_1+b_1i)-(a_2+b_2i) = (a_1-a_2)+(b_1-b_2)i \]

对于复数$z=a+bi$与其共轭复数$\overline{z}=a-bi$,易得
\[ z+\overline{z}=2a=2Re(z), \  z-\overline{z}=2b=2iIm(z) \]

再定义减法为加法的逆运算,那么按照实数的运算定律,有
\[ (a_1+b_1i) \pm (a_2+b_2i) = (a_1 \pm a_2) + (b_1 \pm b_2)i \]
这显然也可以推广到任意有限个复数相加的情形,显然,复数的加减法对应着数对的加减法,也就对应着向量的加减法。

不难验证,关于共轭复数有
\[ \overline{z_1+z_2}=\overline{z_1}+\overline{z_2} \]

关于乘法,有
\begin{eqnarray*}
  (a_1+b_1i)(a_2+b_2i) & = & a_1a_2+(a_1b_2+a_2b_1)i + b_1b_2 i^2 \\
  & = & (a_1a_2-b_1b_2) + (a_1b_2+a_2b_1)i
\end{eqnarray*}

将复数$z=a+bi$与其共轭复数$\overline{z}=a-bi$相乘,有
\[ z\overline{z}=(a+bi)(a-bi)=a^2-(bi)^2=a^2+b^2 = |z|^2 \]
即复数的模的平方,等于它与其共轭之乘积(这必然会是个实数).

同样规定除法为乘法的逆运算,有
\begin{eqnarray*}
  \frac{a_1+b_1i}{a_2+b_2i} & = & \frac{(a_1+b_1i)(a_2-b_2i)}{(a_2+b_2i)(a_2-b_2i)} \\
  & = & \frac{(a_1a_2+b_1b_2)+(b_1a_2-a_1b_2)i}{a_2^2+b_2^2}
\end{eqnarray*}

同样可以验证
\[ \overline{z_1z_2} = \overline{z_1} \cdot \overline{z_2}, \  \overline{\left( \frac{z_1}{z_2} \right)} = \frac{\overline{z_1}}{\overline{z_2}} \]

借用除法,可以得到复数$z$的共轭为
\[ \overline{z}=\frac{|z|}{z} \]

复数乘法与除法的公式不便于记忆,因此实际计算时,直接计算反而更加容易. 稍后我们将看到,将复数表达为三角形式后,乘法与除法将是相当的直观和简洁,有着明显的几何意义.

\begin{example}[从数对引入复数]
  我们从另一角度引入复数,实数对应着数轴上的点,属于一维数,我们认为经过推广后的复数为二维数,它与坐标平面上的点$(a,b)$一一对应,即复数$z$就是一个数对$(a,b)$,当$b=0$时,它就是实数,即复数$(a,0)$就是实数$a$,现在定义加法如下: 若$z_1=(a_1,b_1)$,$z_2=(a_2,b_2)$,则
  \[ z_1 + z_2 = (a_1 + a_2, b_1 + b_2) \]
  再定义复数的乘法是
  \[ z_1z_2 = (a_1a_2-b_1b_2,a_1b_2+a_2b_1) \]
  在这定义下,显然加法和乘法都满足交换律,且乘法对于加法的分配律也是容易验证的,此外,容易验证
  \[ (a,b)(1,0) = (a,b) \]
  即复数$(1,0)$(即实数1)在复数范围内仍然是乘法的单位元。还可以验证
  \[ (a,b) = (a,0)(1,0) + (b,0)(0,1)  \]
  即任意实数都经由两个复数单位$(1,0)$与$(0,1)$的实系数线性组合来表出,于是记纵轴上的单位复数$(0,1)$为$i$,则上式可以简写为$(a,b)=a+bi$,这样,我们从数对出发,通过引入加法和乘法的定义也引出了复数的概念。
\end{example}

\subsection{复数的三角形式}
\label{sec:triangle-form-of-complex-number}



利用变换$x=r\cos{\theta}$,$y=r\sin{\theta}$,复数$z=a+bi$可以改写为
\[ z=r(\cos{\theta}+i\sin{\theta}) \]
这称为复数的三角形式,其中$r=\sqrt{x^2+y^2}$为复数的模,即 $r=|z|$,角$\theta$称为这复数的\emph{辐角},记作$Arg(z)$,由于三角函数的周期性,将满足$0\leqslant \theta < 2\pi$的那个辐角称为$z$的\emph{辐角主值},记作$arg(z)$.

显然,如果两个复数相等,当且仅当它们的模相等,并且它们辐角集合相等。

我们看一下在这种形式下复数的乘除法运算:
\begin{eqnarray*}
  &&  r_1(\cos{\theta_1}+i\sin{\theta_1}) \cdot r_2(\cos{\theta_2}+i\sin{\theta_2}) \\
  & = & r_1r_2[(\cos{\theta_1}\cos{\theta_2}-\sin{\theta_1}\sin{\theta_2})+(\cos{\theta_1}\sin{\theta_2}+\cos{\theta_2}\sin{\theta_1})i] \\
  & = & r_1r_2(\cos{(\theta_1+\theta_2)}+i\sin{(\theta_1+\theta_2)})
\end{eqnarray*}
而
\begin{eqnarray*}
  &&  \frac{r_1(\cos{\theta_1}+i\sin{\theta_1})}{r_2(\cos{\theta_2}+i\sin{\theta_2})} \\
  & = & \frac{r_1(\cos{\theta_1}+i\sin{\theta_1}) \cdot r_2(\cos{\theta_2}-i\sin{\theta_2})}{r_2(\cos{\theta_2}+i\sin{\theta_2}) \cdot r_2(\cos{\theta_2}-i\sin{\theta_2})} \\
  & = & \frac{r_1r_2[(\cos{\theta_1}\cos{\theta_2}+\sin{\theta_1}\sin{\theta_2})+i(\sin{\theta_1}\cos{\theta_2}-\cos{\theta_1}\sin{\theta_2})]]}{r_2^2(\cos^2{\theta_2}+\sin^2{\theta_2})} \\
  & = & \frac{r_1}{r_2}[\cos{(\theta_1-\theta_2)+i\sin{(\theta_1-\theta_2)}}]
\end{eqnarray*}

可见两个复数相乘除,就是将两个复数的模相乘除得到乘积或商的模,两个复数的辐角相加减得到乘积或商的辐角,这与向量的乘法(无论内积还是外积)不再一致,复数乘法在三角形式下变得相当简单,而且这显然可以推广到任意有限个复数相乘的情形。

\begin{example}[利用共轭复数证明余弦定理]
  取复数$z_1=r_1(\cos{\theta_1}+i\sin{\theta_1})$,$z_2=r_2(\cos{\theta_2}+i\sin{\theta_2})$,则有
  \[ z_1\overline{z_2} = r_1r_2(\cos{(\theta_1-\theta_2)+i\sin{(\theta_1-\theta_2)}}) \]
  因此,$z_1\overline{z_2}$的实数部分就是$z_1$与$z_2$两个复数对应的两个向量的内积,同理$\overline{z_1}z_2$的实数部分也是这内积,所以这内积等于
  \[ \frac{1}{2}(z_1\overline{z_2}+\overline{z_1}z_2) \]
  由恒等式
  \[ (z_1-z_2)(\overline{z_1}-\overline{z_2})= z_1\overline{z_1} + z_2\overline{z_2} -(z_1\overline{z_2}+\overline{z_1}z_2) \]
  左边就是$|z_1-z_2|^2$,右边前两项分别是$|z_1|^2$和$|z_2|^2$,最后的两项就是$z_1$与$z_2$对应两个向量的内积的2倍,于是便得出余弦定理
  \[ |z_1-z_2|^2 = |z_1|^2+|z_2|^2-2|z_1||z_2|\cos{(\theta_1-\theta_2)} \]
\end{example}

\subsection{复数的乘方与棣莫弗公式}
\label{sec:demoivre-formual}


更特别的是复数的乘幂,容易知道
\begin{equation}
  \label{eq:de-moivre-formula}
  [r(\cos{\theta}+i\sin{\theta})]^n = r^n(\cos{n\theta}+i\sin{n\theta})
\end{equation}
这就是复数乘幂的\emph{棣莫弗公式}.

\begin{example}[正余弦的$n$倍角公式与切比雪夫多项式]
  利用棣莫弗公式,我们可以得到正余弦的$n$倍角公式,在棣莫弗公式中令$r=1$,得
  \[ \cos{n\theta}+i\sin{n\theta} = (\cos{\theta}+i\sin{\theta})^n \]
  将右边利用二项式定理展开,得
  \[ \cos{n\theta}+i\sin{n\theta} = \sum_{k=0}^n C_n^k i^k \cos^{n-k}{\theta}\sin^k{\theta} \]
  当$k$为偶数时,求和中的通项变为实数,当$k$为奇数时,则它为虚数,据此可以将上式右端的实数和虚部分开
  \begin{eqnarray*}
   && \cos{n\theta}+i\sin{n\theta}  \\
    & = & \sum_{0 \leqslant 2r \leqslant n}^n (-1)^rC_n^{2r} \cos^{n-2r}{\theta}\sin^{2r}{\theta} + i\sum_{0 \leqslant 2r+1 \leqslant n}(-1)^rC_n^{2r+1} \cos^{n-2r-1}{\theta}\sin^{2r+1}{\theta} 
  \end{eqnarray*}
  于是得到
  \begin{eqnarray}
    \label{eq:cos-sin-of-n-theta}
    \cos{n\theta} & = & \sum_{0 \leqslant 2r \leqslant n}^n (-1)^rC_n^{2r} \cos^{n-2r}{\theta}\sin^{2r}{\theta} \\
    \sin{n\theta} & = & \sum_{0 \leqslant 2r+1 \leqslant n}(-1)^rC_n^{2r+1} \cos^{n-2r-1}{\theta}\sin^{2r+1}{\theta}
  \end{eqnarray}
  这就是余弦和正弦的$n$倍角公式.

  根据这公式,由于$\cos{n\theta}$的每一项中的正弦的指数都是偶数,所以都可以化为余弦,于是$\cos{n\theta}$可以展开为$\cos{\theta}$的$n$次多项式,这就是\emph{第一类切比雪夫多项式},即
  \[ T_n(x) = \sum_{0 \leqslant 2r\leqslant n}(-1)^rC_n^{2r}x^{n-2r}(1-x^2)^r \]
  在$\sin{n\theta}$的展式中,$\sin{\theta}$的次数都是奇数,所以$\dfrac{\sin{(n+1)\theta}}{\sin{\theta}}$也可以展开为$\cos{\theta}$的$n$次多项式,这就是\emph{第二类切比雪夫多项式},即
  \[ U_n(x)=\sum_{0 \leqslant 2r+1 \leqslant n+1}(-1)^rC_{n+1}^{2r+1}x^{n-2r}(1-x^2)^r \]
  关于切比雪夫多项式的更多讨论参见\cite{elementary-math-notes}.
\end{example}

\begin{example}
  现在来求和下面两个表达式
  \begin{eqnarray*}
    A_n & = & 1 + r\cos{\theta} + r^2\cos{2\theta}+\cdots+r^n\cos{n\theta} \\
    B_n & = & r\sin{\theta} + r^2\sin{2\theta} + \cdots + r^n\sin{n\theta}
  \end{eqnarray*}
  令$z=r(\cos{\theta}+i\sin{\theta})$,则
  \begin{eqnarray*}
   && A_n+iB_n \\ 
   & = & 1 + r(\cos{\theta}+i\sin{\theta})+r^2(\cos{2\theta}+i\sin{2\theta})+\cdots+r^n(\cos{n\theta}+i\sin{n\theta}) \\
   & = & 1+ z + z^2 + \cdots + z^n \\
   & = & \frac{1-z^{n+1}}{1-z} \\
    & = & \frac{1-r^{n+1}(\cos{(n+1)\theta}+i\sin{(n+1)\theta})}{1-r(\cos{\theta}+i\sin{\theta})} \\
       & = & \frac{1-r^{n+1}(\cos{(n+1)\theta}+i\sin{(n+1)\theta})}{1-r(\cos{\theta}+i\sin{\theta})} \cdot \frac{1-r\cos{\theta}+ir\sin{\theta}}{1-r\cos{\theta}+ir\sin{\theta}} \\
   & = & \frac{1-r\cos{\theta}+r^{n+2}\cos{n\theta}-r^{n+1}\cos{(n+1)\theta}}{1-2r\cos{\theta}+r^2} + \\
    && i \frac{r\sin{\theta}+r^{n+2}\sin{n\theta}-r^{n+1}\sin{(n+1)\theta}}{1-2r\cos{\theta}+r^2}
  \end{eqnarray*}
  于是比较实部和虚部可得
  \begin{eqnarray*}
    A_n & = & \frac{1-r\cos{\theta}+r^{n+2}\cos{n\theta}-r^{n+1}\cos{(n+1)\theta}}{1-2r\cos{\theta}+r^2} \\
    B_n & = & \frac{r\sin{\theta}+r^{n+2}\sin{n\theta}-r^{n+1}\sin{(n+1)\theta}}{1-2r\cos{\theta}+r^2}
  \end{eqnarray*}
\end{example}

\subsection{复数的开方与单位根}
\label{sec:n-th-root-of-one}



有了棣莫弗公式,我们来讨论一下复数的开方。

记$z=r(\cos{\theta}+i\sin{\theta})$,现在来求它的$n$次方根,设$z'=r'(cos{\theta'+i\sin{\theta'}})$是它的一个$n$次方根,按棣莫弗公式,应有
\[ z'^n=r'^n(\cos{n\theta'}+i\sin{n\theta'}) \]
因为$z'^n=z$,所以有$r'^n=r$,以及$n\theta'=\theta+2m\pi(m \in Z)$,即
\[ r'=\sqrt[n]{r}, \  \theta'=\frac{\theta+2m\pi}{n}(m=0,1,\ldots,n-1) \]
或者写成
\[ \sqrt[n]{z}=\sqrt[n]{r} \left( \cos{\frac{\theta+2m\pi}{n}}+i\sin{\frac{\theta+2m\pi}{n}} \right), \  (m=0,1,\ldots,n-1) \]
根据周期性,可知$z$的$n$次方根正好有$n$个,它们均匀分布在复平面上以原点为圆心,以$\sqrt[n]{|z|}$为半径的圆上,于是在复数范围内,任何数都可以开$n$次方.


特别的是当$z=1$时,对1进行开$n$次方,因为$1=\cos{0}+i\sin{0}$,于是得它的根
\[ \varepsilon_i = \cos{\frac{2i\pi}{n}}+i\sin{\frac{2i\pi}{n}}, \  i=0,1,\ldots,n-1 \]
显然 $\varepsilon_0=1$,它们均匀分布在单位圆上,辐角依次为 $0$, $\frac{2\pi}{n}$, $2\cdot \frac{2\pi}{n}$, $\ldots$,$(n-1)\cdot \frac{2\pi}{n}$.

这些根$\varepsilon_i(i=0,1,\ldots,n-1)$称为\emph{$n$次单位根},如果记 $\varepsilon=\varepsilon_1$,那么容易发现,这 $n$ 个根实际上就是
\[ 1, \varepsilon, \varepsilon^2, \ldots, \varepsilon^{n-1} \]
而 $\varepsilon_k=\varepsilon^k(k=0,1,\ldots,n-1)$.

接下来讨论下有哪些性质,先给出结论
\begin{property}
  对于 $n$ 次单位 $\varepsilon_i(i=0,1,\ldots,n-1)$ 或者写成 $1,\varepsilon, \varepsilon^2, \ldots, \varepsilon^{n-1}$,有
  \begin{enumerate}
  \item $\varepsilon^n=1=\varepsilon_i^n$
  \item $\varepsilon_k=\varepsilon^k(k=0,1,\ldots,n-1)$
  \item $\varepsilon_{k+l}=\varepsilon_k\varepsilon_l(k,l\in \mathbb{N})$,这里规定 $\varepsilon_{n+k}=\varepsilon_{k}$以解决下标越界.
  \item $\overline{\varepsilon_k}=\varepsilon_{n-k}$
  \item $1+\varepsilon+\varepsilon^2+\cdots+\varepsilon^{n-1}=0$
  \item 有复变量多项式因式分解 $z^n-1=(z-1)(z-\varepsilon)(z-\varepsilon^2)\cdots (z-\varepsilon^{n-1})$
  \item 单位根的幂方和(下式中 $m \in \mathbb{N}$)
    \[
      \sum_{k=0}^{n-1}\varepsilon_k^m =
      \begin{cases}
        0, & n \nmid m, \\
        n, & n \mid m
      \end{cases}
    \]
  \end{enumerate}
\end{property}

\begin{proof}[证明]
  前三条是容易看出的.

  对于第四条,由
  \[ \overline{\varepsilon_k} = \frac{1}{\varepsilon_k} = \frac{\varepsilon_k^n}{\varepsilon^k} = \varepsilon^{n-k} = \varepsilon_{n-k} \]

  对于第五条,由 $\varepsilon^n=1$ 有
  \[ 0 = 1- \varepsilon^n = (1-\varepsilon)(1+\varepsilon+\varepsilon^2+\cdots+\varepsilon^{n-1}) \]
  第一个因式不可能为零,所以第二个因式必然为零(依据是复数三角形式的乘法),即$1+\varepsilon+\varepsilon^2+\cdots+\varepsilon^{n-1}=0$.

  实际上第五条可以由以下两个等式得来
  \begin{eqnarray*}
    \sum_{k=0}^{n-1} \cos k \frac{2\pi}{n} & = & 0 \\
    \sum_{k=0}^{n-1} \sin k \frac{2\pi}{n} & = & 0
  \end{eqnarray*}
  这里只证第一个等式,第二个也是类似的处理方法,对第一个,为便于书写,我们尝试求和 $\cos\theta + \cos 2\theta + \cdots + \cos n \theta$,对它每一项都乘以因式$2\sin \frac{\theta}{2}$,再应用积化和差公式
  \[ 2\cos\alpha\sin\beta = \sin(\alpha+\beta) - \sin(\alpha-\beta) \]
  可得
  \begin{equation*}
    \begin{split}
      & 2(\cos\theta + \cos 2\theta + \cdots + \cos n \theta)\sin \frac{\theta}{2} \\
      = & \left(\sin \frac{3}{2} \theta - \sin  \frac{\theta}{2} \right) + \left(\sin \frac{5}{2}\theta - \sin \frac{3}{2}\theta \right) + \cdots + \left(\sin \left(n+\frac{1}{2}\right)\theta - \sin \left(n-\frac{1}{2}\right)\theta \right) \\
      = & \sin \left(n+\frac{1}{2}\right)\theta - \sin  \frac{\theta}{2}
    \end{split}
  \end{equation*}
  所以
  \[ \cos\theta + \cos 2\theta + \cdots + \cos n \theta = \frac{\sin \left(n+\frac{1}{2}\right)\theta}{2\sin \frac{\theta}{2}} - \frac{1}{2} \]
  上式中,将 $n$ 替换为 $n-1$,将 $\theta$ 以 $\frac{2\pi}{n}$ 取代,即可得
  \[ \sum_{k=1}^{n-1} \cos k \frac{2\pi}{n} = -1 \]
  注意下标是从1开始的,将下标换成从0开始,就得
  \[ \sum_{k=0}^{n-1} \cos k \frac{2\pi}{n} = 0 \]
\end{proof}


%%% Local Variables:
%%% mode: latex
%%% TeX-master: "../../elementary-math-note"
%%% End: